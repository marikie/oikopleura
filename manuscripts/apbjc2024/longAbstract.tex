\documentclass{article}
\usepackage[utf8]{inputenc}

% Language setting
% Replace `english' with e.g. `spanish' to change the document language
\usepackage[english]{babel}

% Set page size and margins
% Replace `letterpaper' with`a4paper' for UK/EU standard size
% \usepackage[a4paper,top=2cm,bottom=2cm,left=3cm,right=3cm,marginparwidth=1.75cm]{geometry}
\usepackage[a4paper,top=2cm,bottom=2cm,left=2cm,right=2cm,marginparwidth=1.75cm]{geometry}
% \usepackage[a4paper, margin=0.8cm]{geometry}

% Useful packages
\usepackage{amsmath, amssymb}
\usepackage{wasysym} % for consistent geometric shapes
\usepackage{graphicx} % for including graphics
\usepackage{subcaption} % for subfigures
\usepackage{caption} % for subfigures with manual captions
\usepackage{xcolor}  % for setting colors
\usepackage{hyperref} % for hyperlinks
\usepackage{indentfirst} % for indenting the first paragraph


% User-defined commands
\newcommand{\quotes}[1]{``#1''} % Quotation marks
\newcommand{\Cir}{\CIRCLE} % Filled Circle
\newcommand{\Tri}{\blacktriangle} % Triangle
\newcommand{\dTri}{\blacktriangledown} % downward triangle
\newcommand{\Squ}{\blacksquare} % Filled Square
\newcommand{\myhline}{\color{gray}\rule{0.86\textwidth}{0.4pt}\vspace{0.25cm}}

\title{Trends and Patterns of Single-Base Substitutions in Evolution: A Comparative Analysis}
\author{Mariko Nakagawa and Martin C. Frith}
\date{}

\begin{document}


\maketitle

\section{Introduction}

Few studies examine single-base substitutions across diverse organisms. Even a recent comprehensive study of vertebrates did not thoroughly investigate single-base substitutions~\cite{bergeron2023evolution}.

\section{Methods}
We obtained reference genomes for three closely related species (\textit{Species A}, \textit{Species B}, and \textit{Species C}) from NCBI (https://www.ncbi.nlm.nih.gov/datasets), with \textit{Species A} serving as the outgroup in the phylogenetic relationship. Pairwise alignments were performed between \textit{Species A} and \textit{Species B}, and between \textit{Species A} and \textit{Species C}. We examined every set of trinucleotides in the alignments to infer substitutions in \textit{Species B} and \textit{Species C} based on the principle of parsimony. The substitution rate was calculated as the percentage of the number of inferred substitutions divided by the number of original trinucleotides. For instance, if a substitution from $\Cir\Tri\Squ$ to $\Cir\dTri\Squ$ was inferred to have occurred in \textit{Species B}, we incremented the count of the substitution $\Cir\Tri\Squ \rightarrow \Cir\dTri\Squ$ for \textit{Species B} and the count of the original trinucleotide $\Cir\Tri\Squ$ for both \textit{Species B} and \textit{Species C}. Additionally, we incremented the count of the original trinucleotide $\Cir\Tri\Squ$ by 1 if the trinucleotide sequence was $\Cir\Tri\Squ$ in all three species.

\section{Preliminary Results}
We often observe high C>T (or G>A) and T>C (or A>G) substitutions, consistent with the concept of "transition bias." Transitions, mutations between purines (A and G) or pyrimidines (C and T), occur more frequently than transversions, which are mutations between purines and pyrimidines, through evolution.

  It is also known that transitions from strong base pairing to weak base pairing (C>T) occur in a wide range of living organisms, and our results also show this. However, it is interesting to note that this is not always the case. \textit{W.~Hederae}, a fungus often isolated from plant-associated low-water-activity environments and exhibiting halotolerance, shows markedly higher T>C than C>T (\autoref{fig:fig1i}).

  Some species show unique trends of substitutions.
  Two algae species, \textit{P.~bovis} and \textit{P.~ciferrii}, show high C>G substitutions (\autoref{fig:fig1l}). They belong to the genus \textit{Prototheca}, which is a unicellular, achlorophyllous, yeast-like algae that can induce protothecosis in humans and animals.
  The animal plankton, \textit{O.~albicans}, shows high T>A and T>G substitutions (\autoref{fig:fig1m}), while \textit{O.~vanhoeffeni} shows high C>A substitutions.

\begin{figure}[h!]
    \centering
    \begin{subfigure}{\textwidth}
      \centering
      \includegraphics[width=\linewidth]{./images/batBro.pdf}
      \caption{\textit{B.~brooksi}}
      \label{fig:fig1a}
    \end{subfigure}\\

    \begin{subfigure}{\textwidth}
      \centering
      \includegraphics[width=\linewidth]{./images/walHed.pdf}
      \caption{\textit{W.~hederae}}
      \label{fig:fig1i}
    \end{subfigure}\\

    \begin{subfigure}{\textwidth}
      \centering
      \includegraphics[width=\linewidth]{./images/proCif.pdf}
      \caption{\textit{P.~ciferrii}}
      \label{fig:fig1l}
    \end{subfigure}\\

    \begin{subfigure}{\textwidth}
      \centering
      \includegraphics[width=\linewidth]{./images/oikAlb.pdf}
      \caption{\textit{O.~albicans}}
      \label{fig:fig1m}
    \end{subfigure}

    \caption{Single-base substitution mutations of various organisms.}
    \label{fig:fig1}
\end{figure}
\bibliographystyle{unsrt}
\bibliography{ref.bib}

\end{document}

