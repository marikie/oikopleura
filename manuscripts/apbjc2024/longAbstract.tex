\documentclass{article}
\usepackage[utf8]{inputenc}

% Language setting
% Replace `english' with e.g. `spanish' to change the document language
\usepackage[english]{babel}

% Set page size and margins
% Replace `letterpaper' with`a4paper' for UK/EU standard size
\usepackage[a4paper,top=2cm,bottom=2cm,left=3cm,right=3cm,marginparwidth=1.75cm]{geometry}

% Useful packages
\usepackage{amsmath}
\usepackage{graphicx}
\usepackage{subcaption}

\begin{document}

\section{Introduction}


\section{Methods}

\section{Results \& Discussion}
\begin{itemize}
  \item We often observe high C>T (or G>A on the other strand) or T>C (or A>G on the other strand) mutations, which aligns with the established understanding of transition mutations. Transitions can easily occur due to the similarity in their chemical structures. \\
    \\
    However, it is interesting to note that C>T and T>C mutations do not always occur at the same frequency. For example, deep-sea mussels (\textit{B. septemdierum} and \textit{B. brooksi}) and shallow-sea mussels (\textit{M. edulis} and \textit{M. galloprovincialis}) show a high frequency of C>T mutations, but not T>C. On the other hand, \textit{W. hederae}, a fungus often isolated from plant-associated low-water-activity environments and exhibiting halotolerance~\cite{janvcivc2016halophily}, shows higher T>C than C>T.
  \item Some species show unique trends of mutations.
    \begin{itemize}
      \item Two fungi species, \textit{P.bovis} and \textit{P.ciferrii}, show high C>G mutations. They belong to the genus \textit{Prototheca}, which is a unicellular, achlorophyllous, yeast-like algae that can induce protothecosis in humans and animals~\cite{hifney2022microbial}.
      \item The animal plankton \textit{O.vanhoeffeni} shows high C>A mutations, while \textit{O.albicans} shows high T>A and T>G mutations.
    \end{itemize}
\end{itemize}

\begin{figure}[h!]
    \centering
    \begin{subfigure}{0.32\textwidth}
      \centering
      \includegraphics[width=\linewidth]{./images/mut3_gigPla_batSep_batBro_20240610_batBro.pdf}
      \caption{\textit{B.brooksi}}
      \label{fig:fig1a}
    \end{subfigure}
    \hfill
    \begin{subfigure}{0.32\textwidth}
      \centering
      \includegraphics[width=\linewidth]{./images/mut3_gigPla_batSep_batBro_20240610_batSep.pdf}
      \caption{\textit{B.septemdierum}}
      \label{fig:fig1b}
    \end{subfigure}
    \hfill
    \begin{subfigure}{0.32\textwidth}
      \centering
      \includegraphics[width=\linewidth]{./images/mut3_mytTro_mytEdu_mytGal_20240428_mytEdu.pdf}
      \caption{\textit{M.edulis}}
      \label{fig:fig1c}
    \end{subfigure}
    \vspace{0.1cm}
    \begin{subfigure}{0.32\textwidth}
      \centering
      \includegraphics[width=\linewidth]{./images/mut3_mytTro_mytEdu_mytGal_20240428_mytGal.pdf}
      \caption{\textit{M.galloprovincialis}}
      \label{fig:fig1d}
    \end{subfigure}
    \hfill
    \begin{subfigure}{0.32\textwidth}
        \centering
        \includegraphics[width=\linewidth]{./images/mut3_aspCos_aspChe_aspCri_20240617_aspChe.pdf}
        \caption{\textit{A.chevalieri}}
        \label{fig:fig1e}
    \end{subfigure}
    \hfill
    \begin{subfigure}{0.32\textwidth}
        \centering
        \includegraphics[width=\linewidth]{./images/mut3_aspCos_aspChe_aspCri_20240617_aspCri.pdf}
        \caption{\textit{A.crinita}}
        \label{fig:fig1f}
    \end{subfigure}
    \vspace{0.1cm}
    \begin{subfigure}{0.32\textwidth}
      \centering
      \includegraphics[width=\linewidth]{./images/mut3_ulvPro_ulvMut_ulvCom_20240610_ulvCom.pdf}
      \caption{\textit{U.compressa}}
      \label{fig:fig1g}
    \end{subfigure}
    \hfill
    \begin{subfigure}{0.32\textwidth}
      \centering
      \includegraphics[width=\linewidth]{./images/mut3_ulvPro_ulvMut_ulvCom_20240610_ulvMut.pdf}
      \caption{\textit{U.mutabilis}}
      \label{fig:fig1h}
    \end{subfigure}
    \hfill
    \begin{subfigure}{0.32\textwidth}
      \centering
      \includegraphics[width=\linewidth]{./images/mut3_walMel_walIch_walHed_20240611_walHed.pdf}
      \caption{\textit{W.hederae}}
      \label{fig:fig1i}
    \end{subfigure}
    \vspace{0.1cm}
    \begin{subfigure}{0.32\textwidth}
      \centering
      \includegraphics[width=\linewidth]{./images/mut3_walMel_walIch_walHed_20240611_walIch.pdf}
      \caption{\textit{W.ichthyophaga}}
      \label{fig:fig1j}
    \end{subfigure}
    \hfill
    \begin{subfigure}{0.32\textwidth}
      \centering
      \includegraphics[width=\linewidth]{./images/mut3_proCut_proCif_proBov_20240423_proBov.pdf}
      \caption{\textit{P.bovis}}
      \label{fig:fig1k}
    \end{subfigure}
    \hfill
    \begin{subfigure}{0.32\textwidth}
      \centering
      \includegraphics[width=\linewidth]{./images/mut3_proCut_proCif_proBov_20240423_proCif.pdf}
      \caption{\textit{P.ciferrii}}
      \label{fig:fig1l}
    \end{subfigure}
    \vspace{0.1cm}
    \begin{subfigure}{0.32\textwidth}
      \centering
      \includegraphics[width=\linewidth]{./images/mut3_oikDio_oikAlb_oikVan_20240415_oikAlb.pdf}
      \caption{\textit{O.albicans}}
      \label{fig:fig1m}
    \end{subfigure}
    \hfill
    \begin{subfigure}{0.32\textwidth}
      \centering
      \includegraphics[width=\linewidth]{./images/mut3_oikDio_oikAlb_oikVan_20240415_oikVan.pdf}
      \caption{\textit{O.vanhoeffeni}}
      \label{fig:fig1n}
    \end{subfigure}
    \caption{Single-base substitution mutations of various organisms.}
    \label{fig:fig1}
\end{figure}
\bibliographystyle{unsrt}
\bibliography{ref.bib}

\end{document}

